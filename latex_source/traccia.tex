\section{Traccia}
\justify

Simulare una chat multiutente basata su IRC. Utilizzare un approccio client/server.
\subsection*{Server:}
\begin{itemize}
    \item Permette agli utenti di connettersi
    \item Mostra agli utenti una serie di possibili canali attivi (identificati con \#)
    \item Rappresenta un gruppo in cui tutti gli utenti connessi possono inviare messaggi visibili a tutti coloro che sono connessi in quel canale
    \item Permette all'utente di cambiare il canale su cui è connesso
    \item Gestisce la collisione tra nomi utenti uguali
    \item Permette a due utenti di parlare in privato
\end{itemize}

\subsection*{Client:}
\begin{itemize}
    \item Si connette ad un server specificando un nome utente, non è richiesta la password
    \item Può richiedere la lista dei canali inviando \\
    Comando: \texttt{/list}
    \item Può connettersi ad un canale \\
    Comando: \texttt{/join \#channel\_name}
    \item Può vedere gli utenti connessi \\
    Comando: \texttt{/users}
    \item Può inviare messaggi \\
    Comando: \texttt{/msg messaggio}
    \item Può inviare un messaggio privato ad un utente \\
    Comando: \texttt{/privmsg nickname messaggio}
    \item Può cambiare il canale su cui è connesso in qualunque momento
\end{itemize}

\subsection*{Implementare l'utente amministratore che può:}
\begin{itemize}
    \item Espellere un utente dal canale \\
    Comando: \texttt{/kick nickname}
    \item Bannare/sbannare un utente dal canale \\
    Comando: \texttt{/ban nickname} \\
    Comando: \texttt{/unban nickname}
    \item Promuovere un utente come moderatore \\
    Comando: \texttt{/promote nickname}
\end{itemize}