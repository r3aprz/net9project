%----------------------------------------------------------------------------------------
%	PACKAGES AND OTHER DOCUMENT CONFIGURATIONS
%----------------------------------------------------------------------------------------

% CI logo colors

\usepackage[italian]{babel} % English language hyphenation

%\usepackage{authblk}} %Added by Eudis

\usepackage{parskip}
\usepackage{titlesec}
\usepackage{ragged2e}
\usepackage{microtype} % Better typography


\usepackage{amsmath,amsfonts,amsthm,bm} % Math packages for equations

\usepackage[svgnames]{xcolor} % Enabling colors by their 'svgnames'
\definecolor{blueCI}{rgb}{0.0, 0.2, 0.4}
\definecolor{blackCI}{rgb}{0.086, 0.13,0.16}

% \usepackage[hang, small, labelfont=bf, up, textfont=it]{caption} % Custom captions under/above tables and figures

\usepackage{booktabs} % Horizontal rules in tables

\usepackage{lastpage} % Used to determine the number of pages in the document (for "Page X of Total")

\usepackage{graphicx} % Required for adding images

\usepackage{enumitem} % Required for customising lists
\setlist{noitemsep} % Remove spacing between bullet/numbered list elements

\usepackage{sectsty} % Enables custom section titles
\allsectionsfont{\color{blackCI}} % Change the font of all section commands (Helvetica)

\usepackage[colorlinks,linkcolor=blueCI,urlcolor=blueCI,citecolor=blueCI]{hyperref}
\urlstyle{same} % same font as text

\usepackage{lipsum}

%----------------------------------------------------------------------------------------
%	MARGINS AND SPACING
%----------------------------------------------------------------------------------------

\usepackage{geometry} % Required for adjusting page dimensions

\geometry{
	top=1cm, % Top margin
	bottom=1.5cm, % Bottom margin
	left=2cm, % Left margin
	right=2cm, % Right margin
	includehead, % Include space for a header
	includefoot, % Include space for a footer
	%showframe, % Uncomment to show how the type block is set on the page
}

\setlength{\columnsep}{7mm} % Column separation width

%----------------------------------------------------------------------------------------
%	FONTS
%----------------------------------------------------------------------------------------

\usepackage[T1]{fontenc} % Output font encoding for international characters
\usepackage[utf8]{inputenc} % Required for inputting international characters

\usepackage[sfdefault,scaled=.85]{FiraSans}
\usepackage{newtxsf}
%----------------------------------------------------------------------------------------
%	HEADERS AND FOOTERS
%----------------------------------------------------------------------------------------

\usepackage{fancyhdr} % Needed to define custom headers/footers
\pagestyle{fancy} % Enables the custom headers/footers

\renewcommand{\headrulewidth}{0.0pt} % No header rule
\renewcommand{\footrulewidth}{0.4pt} % Thin footer rule

\renewcommand{\sectionmark}[1]{\markboth{#1}{}} % Removes the section number from the header when \leftmark is used

%\nouppercase\leftmark % Add this to one of the lines below if you want a section title in the header/footer

% Headers
\lhead{} % Left header
\chead{\textit{\thetitle}} % Center header - currently printing the article title
\rhead{} % Right header

% Footers
\lfoot{} % Left footer
\cfoot{} % Center footer
\rfoot{\footnotesize \thepage}% de \pageref*{LastPage}} % Right footer, "Page 1 of 2"

\fancypagestyle{firstpage}{ % Page style for the first page with the title
	\fancyhf{}
	\renewcommand{\footrulewidth}{0pt} % Suppress footer rule
}

%----------------------------------------------------------------------------------------
%	TITLE SECTION
%----------------------------------------------------------------------------------------

\newcommand{\authorstyle}[1]{\large\color{blackCI}#1} % Authors style (Helvetica)

\newcommand{\institution}[1]{\footnotesize\usefont{OT1}{phv}{m}{sl}\color{blackCI}#1} % Institutions style (Helvetica)

\usepackage{titling} % Allows custom title configuration

\newcommand{\HorRule}{\color{blueCI}\rule{\linewidth}{1pt}} % Defines the horizontal rule around the title

\newcommand{\TITFONT}{20}

\newcommand{\email}[1]{\normalsize\href{mailto:#1}{#1}\par}

\pretitle{
	\centering \vspace{-30pt} % Move the entire title section up
	\centering \HorRule\vspace{10pt} % Horizontal rule before the title
	\centering \fontsize{\TITFONT}{1.2\TITFONT}\usefont{OT1}{phv}{b}{n}\selectfont % Helvetica
	\centering \color{blackCI} % Text colour for the title and author(s)
}

\posttitle{\par\vskip 15pt} % Whitespace under the title

\preauthor{} % Anything that will appear before \author is printed

\postauthor{ % Anything that will appear after \author is printed
        \centering \vspace{6pt} % Space before the rule
        \vspace{2pt} % Space after the title section
    	\par\HorRule \\ % Horizontal rule after the title
    	\vspace{2pt} % Space after the title section
}

%----------------------------------------------------------------------------------------
%	ABSTRACT
%----------------------------------------------------------------------------------------

\usepackage{lettrine} % Package to accentuate the first letter of the text (lettrine)
\usepackage{fix-cm}	% Fixes the height of the lettrine

\newcommand{\initial}[1]{ % Defines the command and style for the lettrine
	\lettrine[lines=2,findent=4pt,nindent=0pt]{% Lettrine takes up 3 lines, the text to the right of it is indented 4pt and further indenting of lines 2+ is stopped
		\color{blackCI}% Lettrine colour
		{#1}% The letter
	}{}%
}

\usepackage{xstring} % Required for string manipulation

\newcommand{\lettrineabstract}[1]{\color{blackCI}
	\StrLeft{#1}{1}[\firstletter] % Capture the first letter of the abstract for the lettrine
	\initial{\firstletter}\textbf{\StrGobbleLeft{#1}{1}} % Print the abstract with the first letter as a lettrine and the rest in bold
}

%----------------------------------------------------------------------------------------
%	BIBLIOGRAPHY
%----------------------------------------------------------------------------------------

\renewcommand{\refname}{\normalsize{Referências}}

\usepackage[backend=biber,style=ieee,natbib=true]{biblatex} 

\addbibresource{refs.bib} % The filename of the bibliography

\usepackage[autostyle=true]{csquotes} % Required to generate language-dependent quotes in the bibliography

\usepackage{float} % Force the figure to stay where it was determined

\usepackage{nomencl} % List of symbols

\usepackage{listings}